\documentclass[11pt,a4paper]{article}
\usepackage[utf8]{inputenc}
\usepackage[english]{babel}
\usepackage{geometry}
\usepackage{fancyhdr}
\usepackage{graphicx}
\usepackage{amsmath}
\usepackage{amsfonts}
\usepackage{amssymb}
\usepackage{listings}
\usepackage{xcolor}
\usepackage{hyperref}
\usepackage{booktabs}
\usepackage{longtable}
\usepackage{array}
\usepackage{tikz}
\usepackage{enumitem}
\usepackage{float}
\usepackage{caption}
\usepackage{subcaption}
\usepackage{textcomp}
\usepackage{verbatim}

% Page setup
\geometry{margin=1in}
\pagestyle{fancy}
\fancyhf{}
\rhead{Lubrication Feeder Documentation}
\lhead{\leftmark}
\cfoot{\thepage}

% Code listing setup
\lstset{
    basicstyle=\ttfamily\footnotesize,
    backgroundcolor=\color{gray!10},
    frame=single,
    breaklines=true,
    captionpos=b,
    numbers=left,
    numberstyle=\tiny\color{gray},
    keywordstyle=\color{blue},
    commentstyle=\color{green!60!black},
    stringstyle=\color{red}
}

% Hyperref setup
\hypersetup{
    colorlinks=true,
    linkcolor=blue,
    filecolor=magenta,
    urlcolor=cyan,
    pdftitle={Lubrication Feeder Valve System Documentation},
    pdfauthor={Lubrication System Documentation Team}
}

% Custom commands
\newcommand{\code}[1]{\texttt{#1}}
\newcommand{\warning}[1]{\textcolor{red}{\textbf{⚠️ #1}}}
\newcommand{\note}[1]{\textcolor{blue}{\textbf{Note:} #1}}

\title{\Huge\textbf{Lubrication Feeder Valve System}\\\Large Documentation}
\author{Lubrication System Documentation Team}
\date{\today}

\begin{document}

\maketitle
\thispagestyle{empty}

\newpage
\tableofcontents
\newpage

\section{Introduction}

\subsection{What are Valves?}
Valves are mechanical devices that control the flow of fluids (liquids, gases, or slurries) by opening, closing, or partially obstructing various passageways. In our lubrication feeder system, valves are essential components that precisely control the distribution of lubricant to different parts of machinery.

\subsection{Purpose in Lubrication Systems}
The valve system in this project serves several critical functions:
\begin{itemize}
    \item \textbf{Precise Fluid Control}: Regulates the exact amount of lubricant delivered
    \item \textbf{Timing Control}: Ensures lubricant is delivered at the right moment
    \item \textbf{System Protection}: Prevents over-lubrication and contamination
    \item \textbf{Automation}: Enables automated lubrication cycles without human intervention
\end{itemize}

\section{Valve Components and Purpose}

\subsection{Hardware Components}

\subsubsection{Valve 1 (Primary Lubrication Valve)}
\begin{itemize}
    \item \textbf{Location}: Connected to GPIO Pin 27
    \item \textbf{Function}: Controls primary lubricant flow
    \item \textbf{Type}: Solenoid-operated valve
    \item \textbf{Default State}: Normally closed (LOW)
\end{itemize}

\subsubsection{Valve 2 (Secondary Lubrication Valve)}
\begin{itemize}
    \item \textbf{Location}: Connected to GPIO Pin 32
    \item \textbf{Function}: Controls secondary lubricant flow or backup system
    \item \textbf{Type}: Solenoid-operated valve
    \item \textbf{Default State}: Normally closed (LOW)
\end{itemize}

\subsubsection{Linear Potentiometer (SoftPot)}
\begin{itemize}
    \item \textbf{Location}: Connected to GPIO Pin 33 (Analog Input)
    \item \textbf{Model}: Spectra Symbol SoftPot (1.969 inch variant with male pin connector)
    \item \textbf{Function}: Provides position feedback for conditional actuator control
    \item \textbf{Accuracy}: ±1\% linearity
    \item \textbf{Activation Threshold}: >100 (analog reading)
    \item \textbf{Purpose}: Enables conditional dual-movement functionality
\end{itemize}

\subsubsection{IR Break Beam Sensor (Optical Sensor)}
\begin{itemize}
    \item \textbf{Location}: Connected to GPIO Pin 16 (Digital Input)
    \item \textbf{Model}: IR Break Beam Sensors with Premium Wire Header Ends - 3mm LEDs
    \item \textbf{Function}: Detects hose presence and position
    \item \textbf{Operation}: Active LOW (beam broken = LOW signal = hose detected)
    \item \textbf{Configuration}: INPUT\_PULLUP (internal pull-up resistor enabled)
    \item \textbf{Purpose}: Provides feedback on hose positioning for quality control
\end{itemize}

\subsubsection{Alcohol Level Sensor 1 (2/3 Tank Level)}
\begin{itemize}
    \item \textbf{Location}: Connected to GPIO Pin 39 (Digital Input)
    \item \textbf{Function}: Detects alcohol level at 2/3 tank capacity
    \item \textbf{Type}: Digital level sensor
    \item \textbf{Operation}: Active HIGH (HIGH = liquid present, LOW = no liquid)
    \item \textbf{Configuration}: INPUT (standard digital input)
    \item \textbf{Purpose}: Monitors upper tank level for full/medium detection
\end{itemize}

\subsubsection{Alcohol Level Sensor 2 (1/3 Tank Level)}
\begin{itemize}
    \item \textbf{Location}: Connected to GPIO Pin 36 (Digital Input)
    \item \textbf{Function}: Detects alcohol level at 1/3 tank capacity
    \item \textbf{Type}: Digital level sensor
    \item \textbf{Operation}: Active HIGH (HIGH = liquid present, LOW = no liquid)
    \item \textbf{Configuration}: INPUT (standard digital input)
    \item \textbf{Purpose}: Monitors lower tank level for medium/empty detection
\end{itemize}

\subsubsection{Hose Holder Servo}
\begin{itemize}
    \item \textbf{Location}: Connected to GPIO Pin 14 (PWM Output)
    \item \textbf{Function}: Controls hose holder position (open/close)
    \item \textbf{Type}: Standard servo motor (50Hz PWM)
    \item \textbf{Operation}: 0° = closed position, 90° = open position
    \item \textbf{Configuration}: PWM output with ESP32Servo library
    \item \textbf{Purpose}: Automated hose positioning and securing mechanism
\end{itemize}

\subsubsection{Electromagnet Control}
\begin{itemize}
    \item \textbf{Location}: Connected to GPIO Pin 17 (Digital Output)
    \item \textbf{Function}: Controls electromagnet via MOSFET for attachment detection
    \item \textbf{Type}: Digital control signal to MOSFET driver
    \item \textbf{Operation}: HIGH = electromagnet ON, LOW = electromagnet OFF
    \item \textbf{Configuration}: OUTPUT (digital output pin)
    \item \textbf{Purpose}: Electromagnet control and attachment status verification
\end{itemize}

\subsection{Control System Components}

\begin{figure}[H]
\centering
\begin{tikzpicture}[node distance=2cm, auto]
    % ESP32 MCU
    \node [draw, rectangle, minimum width=3cm, minimum height=4cm] (esp32) {ESP32 MCU};
    
    % GPIO Pins
    \node [draw, rectangle, right=of esp32, minimum width=2.5cm, minimum height=4cm] (gpio) {GPIO Pins};
    
    % Components
    \node [draw, rectangle, right=of gpio, yshift=1.5cm] (valve1) {Valve 1};
    \node [draw, rectangle, right=of gpio, yshift=1cm] (valve2) {Valve 2};
    \node [draw, rectangle, right=of gpio, yshift=0.5cm] (pot) {Potentiometer};
    \node [draw, rectangle, right=of gpio, yshift=0cm] (ir) {IR Sensor};
    \node [draw, rectangle, right=of gpio, yshift=-0.5cm] (level1) {Level Sensor 1};
    \node [draw, rectangle, right=of gpio, yshift=-1cm] (level2) {Level Sensor 2};
    \node [draw, rectangle, right=of gpio, yshift=-1.5cm] (servo) {Hose Holder};
    \node [draw, rectangle, right=of gpio, yshift=-2cm] (magnet) {Electromagnet};
    
    % CAN Bus
    \node [draw, rectangle, below=of esp32, yshift=-1cm] (can) {CAN Bus Communication};
    
    % Lubricant Flow
    \node [draw, rectangle, below right=of gpio, yshift=-1cm] (flow) {Lubricant Flow};
    
    % Arrows
    \draw [->] (esp32) -- (gpio);
    \draw [->] (gpio) -- (valve1) node[midway, above] {GPIO 27};
    \draw [->] (gpio) -- (valve2) node[midway, above] {GPIO 32};
    \draw [<-] (gpio) -- (pot) node[midway, above] {GPIO 33};
    \draw [<-] (gpio) -- (ir) node[midway, above] {GPIO 16};
    \draw [<-] (gpio) -- (level1) node[midway, above] {GPIO 39};
    \draw [<-] (gpio) -- (level2) node[midway, above] {GPIO 36};
    \draw [->] (gpio) -- (servo) node[midway, above] {GPIO 14};
    \draw [->] (gpio) -- (magnet) node[midway, above] {GPIO 17};
    \draw [->] (esp32) -- (can);
    \draw [->] (gpio) -- (flow);
\end{tikzpicture}
\caption{Control System Architecture}
\label{fig:control_system}
\end{figure}

\section{How Valves Operate}

\subsection{Basic Operation Principle}

\begin{enumerate}
    \item \textbf{Electrical Signal}: The ESP32 microcontroller sends a HIGH signal to the valve's GPIO pin
    \item \textbf{Solenoid Activation}: The electrical signal energizes the solenoid coil
    \item \textbf{Mechanical Movement}: The energized coil creates a magnetic field that moves the valve plunger
    \item \textbf{Flow Control}: The plunger movement opens the valve, allowing lubricant to flow
    \item \textbf{Deactivation}: When the signal goes LOW, the spring returns the plunger to the closed position
\end{enumerate}

\subsection{Timing Control}
Each valve can be activated for a specific duration:
\begin{itemize}
    \item \textbf{Minimum Duration}: 1ms (theoretical)
    \item \textbf{Default Duration}: 1000ms (1 second)
    \item \textbf{Maximum Duration}: 65,535ms (approximately 65 seconds)
    \item \textbf{Custom Duration}: Configurable via CAN bus commands
\end{itemize}

\subsection{Valve State Diagram}

\begin{figure}[H]
\centering
\begin{tikzpicture}[node distance=3cm, auto]
    \node [draw, circle, minimum size=2cm] (closed1) {CLOSED\\(LOW/0V)};
    \node [draw, circle, right=of closed1, minimum size=2cm] (open) {OPEN\\(HIGH/5V)};
    \node [draw, circle, below=of open, minimum size=2cm] (closed2) {CLOSED\\(LOW/0V)};
    
    \draw [->] (closed1) -- (open) node[midway, above] {Activation Command};
    \draw [->] (open) -- (closed2) node[midway, right] {Timer Expires};
    \draw [->] (closed2) -- (closed1) node[midway, left] {Default State};
\end{tikzpicture}
\caption{Valve State Transition Diagram}
\label{fig:valve_states}
\end{figure}

\section{System Architecture}

\subsection{Communication Protocol}
The valve system uses CAN bus communication for control:

\begin{itemize}
    \item \textbf{Device CAN ID}: 0x191 (receives commands)
    \item \textbf{Response CAN ID}: 0x591 (sends status)
    \item \textbf{Command Format}: 8-byte CAN frames
\end{itemize}

\subsubsection{Acknowledgment Messages}
All commands now send an acknowledgment message before executing their main functionality:
\begin{itemize}
    \item \textbf{Format}: [0xAA, function\_code, 0x00, 0x00, 0x00, 0x00, 0x00, 0x00]
    \item \textbf{Purpose}: Confirms command receipt and indicates start of execution
    \item \textbf{Timing}: Sent immediately upon receiving a valid command
    \item \textbf{Example}: For command 0x03 (Valve 1), acknowledgment is [0xAA, 0x03, 0x00, 0x00, 0x00, 0x00, 0x00, 0x00]
\end{itemize}

\note{The \code{enableAckMessages} variable can be set to \code{false} to disable acknowledgment messages during debugging. This is useful when you want to reduce CAN bus traffic or focus on main command responses only. Set \code{enableAckMessages = true} (default) to enable acknowledgments, or \code{enableAckMessages = false} to disable them.}

\subsubsection{Status Codes}
\begin{itemize}
    \item 0x01: Success
    \item 0x02: Failure
    \item 0x03: Timeout
    \item 0x04: Network/Communication Error
    \item 0xFF: Unknown Command
\end{itemize}

\subsection{Memory Management}
The system tracks valve usage with persistent counters:
\begin{itemize}
    \item \textbf{EEPROM Storage}: Counters saved to flash memory
    \item \textbf{Valve 1 Counter}: Address 12 in EEPROM
    \item \textbf{Valve 2 Counter}: Address 16 in EEPROM
    \item \textbf{Auto-increment}: Counters increase with each activation
\end{itemize}

\section{Operation Instructions}

\subsection{Basic Valve Activation}

\subsubsection{Method 1: Using CAN Commands}

\textbf{Activate Valve 1:}
\begin{lstlisting}[language=C, caption=Valve 1 Activation Command]
CAN ID: 0x191
Command: 0x03
Data: [0x03, duration_high, duration_low, 0x00, 0x00, 0x00, 0x00, 0x00]
\end{lstlisting}

\textbf{Activate Valve 2:}
\begin{lstlisting}[language=C, caption=Valve 2 Activation Command]
CAN ID: 0x191
Command: 0x04
Data: [0x04, duration_high, duration_low, 0x00, 0x00, 0x00, 0x00, 0x00]
\end{lstlisting}

\subsubsection{Method 2: Default Duration (1 second)}
\begin{lstlisting}[language=C, caption=Default Duration Commands]
CAN ID: 0x191
Data: [0x03, 0x00, 0x00, 0x00, 0x00, 0x00, 0x00, 0x00]  // Valve 1
Data: [0x04, 0x00, 0x00, 0x00, 0x00, 0x00, 0x00, 0x00]  // Valve 2
\end{lstlisting}

\subsection{Counter Management}

\textbf{Read Valve Counters:}
\begin{itemize}
    \item Valve 1 Counter: Send command 0x0A
    \item Valve 2 Counter: Send command 0x0B
\end{itemize}

\textbf{Reset Valve Counters:}
\begin{itemize}
    \item Reset Valve 1: Send command 0x0C
    \item Reset Valve 2: Send command 0x0D
\end{itemize}

\subsection{Step-by-Step Operation Example}

\begin{enumerate}
    \item \textbf{System Initialization}
    \begin{itemize}
        \item Power on the ESP32 controller
        \item Wait for "CAN system ready" message
        \item Verify both CAN buses are operational
    \end{itemize}
    
    \item \textbf{Valve Activation Sequence}
    \begin{enumerate}
        \item Send activation command via CAN
        \item System processes command
        \item GPIO pin goes HIGH
        \item Valve opens for specified duration
        \item GPIO pin returns to LOW
        \item Valve closes
        \item Counter increments
        \item Status response sent
    \end{enumerate}
    
    \item \textbf{Monitoring and Verification}
    \begin{itemize}
        \item Check serial monitor for confirmation messages
        \item Verify valve counter incrementation
        \item Monitor CAN bus for status responses
    \end{itemize}
\end{enumerate}

\section{Safety Precautions}

\subsection{\warning{Critical Safety Guidelines}}

\subsubsection{Electrical Safety}
\begin{itemize}
    \item \textbf{Power Isolation}: Always disconnect power before maintenance
    \item \textbf{Voltage Verification}: Ensure 5V supply is stable and clean
    \item \textbf{Ground Connection}: Verify proper system grounding
    \item \textbf{Wire Inspection}: Check for damaged or exposed wiring
\end{itemize}

\subsubsection{Fluid Safety}
\begin{itemize}
    \item \textbf{Lubricant Compatibility}: Use only approved lubricants
    \item \textbf{Pressure Limits}: Do not exceed maximum system pressure (check valve specifications)
    \item \textbf{Leak Prevention}: Inspect all connections for leaks before operation
    \item \textbf{Containment}: Have spill containment measures in place
\end{itemize}

\subsubsection{Operational Safety}
\begin{itemize}
    \item \textbf{Emergency Stop}: Implement emergency shutdown procedures
    \item \textbf{Pressure Relief}: Ensure pressure relief valves are functional
    \item \textbf{Personal Protective Equipment}: Wear appropriate PPE when handling lubricants
    \item \textbf{Ventilation}: Ensure adequate ventilation in enclosed spaces
\end{itemize}

\subsection{Best Practices}

\subsubsection{Pre-Operation Checklist}
\begin{itemize}[label=$\square$]
    \item Verify power supply stability
    \item Check all electrical connections
    \item Inspect valve mounting and alignment
    \item Test emergency stop functionality
    \item Verify lubricant levels and quality
\end{itemize}

\subsubsection{During Operation}
\begin{itemize}
    \item Monitor system pressure continuously
    \item Watch for unusual noises or vibrations
    \item Check for leaks at regular intervals
    \item Maintain communication with control system
\end{itemize}

\subsubsection{Post-Operation}
\begin{itemize}
    \item Record operation hours and cycles
    \item Check valve counter readings
    \item Inspect for wear or damage
    \item Clean external surfaces
\end{itemize}

\section{Troubleshooting Guide}

\subsection{Common Issues and Solutions}

\subsubsection{Issue 1: Valve Not Responding}
\textbf{Symptoms:}
\begin{itemize}
    \item No valve activation despite command
    \item GPIO pin remains LOW
    \item No counter increment
\end{itemize}

\textbf{Possible Causes \& Solutions:}
\begin{enumerate}
    \item \textbf{Power Supply Issue}
    \begin{itemize}
        \item Check 5V supply voltage
        \item Verify current capacity (minimum 500mA per valve)
        \item Solution: Replace or upgrade power supply
    \end{itemize}
    
    \item \textbf{Wiring Problem}
    \begin{itemize}
        \item Inspect connections to GPIO pins 27 and 32
        \item Check for loose or corroded connections
        \item Solution: Clean and tighten connections
    \end{itemize}
    
    \item \textbf{Software Issue}
    \begin{itemize}
        \item Verify correct CAN ID (0x191)
        \item Check command format
        \item Solution: Review command structure and timing
    \end{itemize}
\end{enumerate}

\subsubsection{Issue 2: Valve Stuck Open}
\textbf{Symptoms:}
\begin{itemize}
    \item Continuous lubricant flow
    \item GPIO pin stuck HIGH
    \item System unresponsive
\end{itemize}

\textbf{Solutions:}
\begin{enumerate}
    \item \textbf{Immediate Action}: Cut power to system
    \item \textbf{Check}: Solenoid coil for overheating
    \item \textbf{Inspect}: Valve plunger for mechanical obstruction
    \item \textbf{Replace}: Faulty solenoid if necessary
\end{enumerate}

\subsubsection{Issue 3: Inconsistent Timing}
\textbf{Symptoms:}
\begin{itemize}
    \item Valve duration doesn't match command
    \item Erratic opening/closing behavior
\end{itemize}

\textbf{Solutions:}
\begin{enumerate}
    \item \textbf{Check}: System clock accuracy
    \item \textbf{Verify}: Timer interrupt functionality
    \item \textbf{Update}: Firmware if necessary
    \item \textbf{Calibrate}: Timing parameters
\end{enumerate}

\subsubsection{Issue 4: Communication Errors}
\textbf{Symptoms:}
\begin{itemize}
    \item CAN bus timeout errors
    \item Status code 0x04 (Network Error)
    \item No response to commands
\end{itemize}

\textbf{Solutions:}
\begin{enumerate}
    \item \textbf{Check}: CAN bus wiring and termination
    \item \textbf{Verify}: Baud rate settings (125 kbps)
    \item \textbf{Inspect}: MCP2515 module connections
    \item \textbf{Test}: With CAN bus analyzer tool
\end{enumerate}

\subsection{Diagnostic Commands}

\textbf{System Health Check:}
\begin{lstlisting}[language=C, caption=System Health Check]
Command 0x02: Heartbeat (verifies system responsiveness)
Expected Response: [0x02, 0x01, 0x00, 0x00, 0x00, 0x00, 0x00, 0x00]
\end{lstlisting}

\textbf{Counter Verification:}
\begin{lstlisting}[language=C, caption=Counter Verification]
Command 0x0A: Read Valve 1 Counter
Command 0x0B: Read Valve 2 Counter
\end{lstlisting}

\section{Advanced Movement Control}

\note{All commands listed below send an acknowledgment message (0xAA + function\_code) immediately upon receipt, followed by the main response after execution.}

\subsection{Command Reference}

\subsubsection{Command 0x05: Move Feeder (Speed Mode)}
\begin{itemize}
    \item \textbf{Format}: [0x05, speed\_high, speed\_low, direction, acceleration, 0x00, 0x00, 0x00]
    \item \textbf{Function}: Moves main feeder using speed mode with specified speed, direction, and acceleration
    \item \textbf{Response}: [0x05, status, 0x00, 0x00, 0x00, 0x00, 0x00, 0x00]
\end{itemize}

\subsubsection{Command 0x13: Move Pre-feeder (Speed Mode)}
\begin{itemize}
    \item \textbf{Format}: [0x13, speed\_high, speed\_low, direction, acceleration, 0x00, 0x00, 0x00]
    \item \textbf{Function}: Moves pre-feeder using speed mode with specified speed, direction, and acceleration
    \item \textbf{Response}: [0x13, status, 0x00, 0x00, 0x00, 0x00, 0x00, 0x00]
\end{itemize}

\subsubsection{Command 0x14: Conditional Dual Movement}
\begin{itemize}
    \item \textbf{Format}: [0x14, speed\_high, speed\_low, direction, acceleration, 0x00, 0x00, 0x00]
    \item \textbf{Function}: Moves feeder in speed mode; also moves pre-feeder if linear potentiometer is activated (>100). Pre-feeder direction is determined by potentiometer value: >512 = forward, 100-512 = reverse
    \item \textbf{Response}: [0x14, feeder\_status, prefeeder\_status, 0x00, 0x00, 0x00, 0x00, 0x00]
\end{itemize}

\subsubsection{Command 0x15: Check Hose Position}
\begin{itemize}
    \item \textbf{Format}: [0x15, 0x00, 0x00, 0x00, 0x00, 0x00, 0x00, 0x00]
    \item \textbf{Function}: Checks if hose is in position using IR break beam sensor
    \item \textbf{Response}: [0x15, 0x01, hose\_status, 0x00, 0x00, 0x00, 0x00, 0x00]
    \item \textbf{hose\_status}: 0x01 = hose in position (beam broken), 0x00 = hose not in position (beam intact)
\end{itemize}

\subsubsection{Command 0x16: Check Alcohol Level}
\begin{itemize}
    \item \textbf{Format}: [0x16, 0x00, 0x00, 0x00, 0x00, 0x00, 0x00, 0x00]
    \item \textbf{Function}: Monitors alcohol tank level using two digital level sensors at 2/3 and 1/3 positions
    \item \textbf{Response}: [0x16, 0x01, level\_status, sensor1\_state, sensor2\_state, 0x00, 0x00, 0x00]
    \item \textbf{level\_status}: 
    \begin{itemize}
        \item 0x03 = FULL (both sensors detect liquid)
        \item 0x02 = MEDIUM (only 1/3 sensor detects liquid)
        \item 0x00 = EMPTY (no sensors detect liquid)
        \item 0xFF = ERROR (invalid sensor state)
    \end{itemize}
    \item \textbf{sensor1\_state}: 0x01 = liquid present at 2/3 level, 0x00 = no liquid at 2/3 level
    \item \textbf{sensor2\_state}: 0x01 = liquid present at 1/3 level, 0x00 = no liquid at 1/3 level
\end{itemize}

\subsubsection{Command 0x17: Open Hose Holder}
\begin{itemize}
    \item \textbf{Format}: [0x17, 0x00, 0x00, 0x00, 0x00, 0x00, 0x00, 0x00]
    \item \textbf{Function}: Opens the hose holder servo to 90 degrees (open position)
    \item \textbf{Response}: [0x17, 0x01, 0x00, 0x00, 0x00, 0x00, 0x00, 0x00]
\end{itemize}

\subsubsection{Command 0x18: Close Hose Holder}
\begin{itemize}
    \item \textbf{Format}: [0x18, 0x00, 0x00, 0x00, 0x00, 0x00, 0x00, 0x00]
    \item \textbf{Function}: Closes the hose holder servo to 0 degrees (closed position)
    \item \textbf{Response}: [0x18, 0x01, 0x00, 0x00, 0x00, 0x00, 0x00, 0x00]
\end{itemize}

\subsubsection{Command 0x19: Attach Electromagnet}
\begin{itemize}
    \item \textbf{Format}: [0x19, 0x00, 0x00, 0x00, 0x00, 0x00, 0x00, 0x00]
    \item \textbf{Function}: Activates electromagnet (HIGH)
    \item \textbf{Response}: [0x19, 0x01, 0x00, 0x00, 0x00, 0x00, 0x00, 0x00]
\end{itemize}

\subsubsection{Command 0x1A: Detach Electromagnet}
\begin{itemize}
    \item \textbf{Format}: [0x1A, 0x00, 0x00, 0x00, 0x00, 0x00, 0x00, 0x00]
    \item \textbf{Function}: Deactivates electromagnet (LOW)
    \item \textbf{Response}: [0x1A, 0x01, 0x00, 0x00, 0x00, 0x00, 0x00, 0x00]
\end{itemize}

\subsubsection{Command 0x1B: Read Servo Counter}
\begin{itemize}
    \item \textbf{Format}: [0x1B, 0x00, 0x00, 0x00, 0x00, 0x00, 0x00, 0x00]
    \item \textbf{Function}: Returns current servo counter value
    \item \textbf{Response}: [0x1B, 0x01, counter\_high, counter\_low, 0x00, 0x00, 0x00, 0x00]
    \item \textbf{counter\_high}: High byte of 16-bit counter value
    \item \textbf{counter\_low}: Low byte of 16-bit counter value
\end{itemize}

\subsubsection{Command 0x1C: Reset Servo Counter}
\begin{itemize}
    \item \textbf{Format}: [0x1C, 0x00, 0x00, 0x00, 0x00, 0x00, 0x00, 0x00]
    \item \textbf{Function}: Resets servo counter to zero and saves to EEPROM
    \item \textbf{Response}: [0x1C, 0x01, 0x00, 0x00, 0x00, 0x00, 0x00, 0x00]
\end{itemize}

\subsubsection{Command 0x1D: Acknowledge Message}
\begin{itemize}
    \item \textbf{Format}: [0x1D, 0x00, 0x00, 0x00, 0x00, 0x00, 0x00, 0x00]
    \item \textbf{Function}: Sends acknowledge response with 0xAA in first byte and received function code in second byte
    \item \textbf{Response}: [0xAA, function\_code, 0x00, 0x00, 0x00, 0x00, 0x00, 0x00]
    \item \textbf{function\_code}: The original command code that was received (0x1D in this case)
\end{itemize}

\subsection{Parameters}
\begin{itemize}
    \item \textbf{speed\_high, speed\_low}: 16-bit speed value (0-65535)
    \item \textbf{direction}: 0 = forward, 1 = reverse
    \item \textbf{acceleration}: 8-bit acceleration value (0-255, default 50 if 0)
    \item \textbf{counter\_high, counter\_low}: 16-bit counter value
    \item \textbf{function\_code}: Original command code being acknowledged
\end{itemize}

\section{Maintenance Guidelines}

\subsection{Daily Maintenance (5 minutes)}
\begin{itemize}[label=$\square$]
    \item Visual inspection for leaks
    \item Check system pressure readings
    \item Verify valve counter increments
    \item Monitor serial output for errors
\end{itemize}

\subsection{Weekly Maintenance (15 minutes)}
\begin{itemize}[label=$\square$]
    \item Clean valve exterior surfaces
    \item Check electrical connections
    \item Test emergency stop function
    \item Backup EEPROM counter data
    \item Inspect lubricant quality and levels
\end{itemize}

\subsection{Monthly Maintenance (30 minutes)}
\begin{itemize}[label=$\square$]
    \item Calibrate valve timing
    \item Test all CAN bus commands
    \item Inspect wiring for wear
    \item Update maintenance log
    \item Performance analysis review
\end{itemize}

\subsection{Quarterly Maintenance (1 hour)}
\begin{itemize}[label=$\square$]
    \item Disassemble and clean valve internals
    \item Replace O-rings and seals
    \item Lubricate moving parts
    \item Electrical continuity testing
    \item Firmware update check
    \item System performance optimization
\end{itemize}

\subsection{Annual Maintenance (2 hours)}
\begin{itemize}[label=$\square$]
    \item Complete valve overhaul
    \item Replace wear components
    \item Pressure test entire system
    \item Calibrate all sensors
    \item Update documentation
    \item Training refresh for operators
\end{itemize}

\subsection{Maintenance Schedule Template}

\begin{longtable}{|p{3cm}|p{2cm}|p{2cm}|p{2cm}|p{4cm}|}
\hline
\textbf{Task} & \textbf{Frequency} & \textbf{Last Done} & \textbf{Next Due} & \textbf{Notes} \\
\hline
\endhead
Visual Inspection & Daily & & & \\
\hline
Leak Check & Daily & & & \\
\hline
Electrical Check & Weekly & & & \\
\hline
Valve Cleaning & Monthly & & & \\
\hline
Seal Replacement & Quarterly & & & \\
\hline
Complete Overhaul & Annual & & & \\
\hline
\end{longtable}

\section{Technical Specifications}

\subsection{Electrical Specifications}
\begin{itemize}
    \item \textbf{Operating Voltage}: 5V DC
    \item \textbf{Current Consumption}: 200-500mA per valve
    \item \textbf{Control Signal}: Digital GPIO (0V/5V)
    \item \textbf{Response Time}: <10ms
    \item \textbf{Duty Cycle}: 100\% (continuous operation capable)
\end{itemize}

\subsection{Mechanical Specifications}
\begin{itemize}
    \item \textbf{Operating Pressure}: 0-10 bar (check valve datasheet)
    \item \textbf{Temperature Range}: -10°C to +60°C
    \item \textbf{Flow Rate}: Depends on valve size and pressure
    \item \textbf{Connection Type}: Standard pipe fittings
    \item \textbf{Mounting}: Panel or manifold mount
\end{itemize}

\subsection{Communication Specifications}
\begin{itemize}
    \item \textbf{Protocol}: CAN 2.0B
    \item \textbf{Baud Rate}: 125 kbps
    \item \textbf{Message Format}: Standard 11-bit identifier
    \item \textbf{Data Length}: 8 bytes
    \item \textbf{Error Detection}: CRC and acknowledgment
\end{itemize}

\section{Real-World Applications}

\subsection{Industrial Automation}
\textbf{Scenario}: Automated assembly line lubrication
\begin{itemize}
    \item \textbf{Application}: Precise lubricant application to moving parts
    \item \textbf{Benefits}: Consistent lubrication, reduced waste, improved reliability
    \item \textbf{Example}: Automotive manufacturing conveyor systems
\end{itemize}

\subsection{Maintenance Systems}
\textbf{Scenario}: Scheduled equipment lubrication
\begin{itemize}
    \item \textbf{Application}: Timed lubrication of bearings and gears
    \item \textbf{Benefits}: Extended equipment life, reduced downtime
    \item \textbf{Example}: Wind turbine gearbox lubrication
\end{itemize}

\subsection{Quality Control}
\textbf{Scenario}: Controlled lubricant application in testing
\begin{itemize}
    \item \textbf{Application}: Precise amounts for product testing
    \item \textbf{Benefits}: Repeatable test conditions, accurate results
    \item \textbf{Example}: Bearing life testing laboratories
\end{itemize}

\subsection{Research and Development}
\textbf{Scenario}: Experimental lubrication studies
\begin{itemize}
    \item \textbf{Application}: Variable timing and quantity testing
    \item \textbf{Benefits}: Data collection, parameter optimization
    \item \textbf{Example}: Tribology research facilities
\end{itemize}

\section{Conclusion}

This valve system provides precise, automated control of lubricant flow with comprehensive monitoring and safety features. Regular maintenance and proper operation ensure reliable performance and extended system life.

For technical support or additional information, refer to the source code comments in \code{lubrication\_feeder.ino} or contact the system administrator.

\vspace{1cm}
\hrule
\vspace{0.5cm}

\noindent\textbf{Document Version}: 1.0 \\
\textbf{Last Updated}: \today \\
\textbf{Author}: Lubrication System Documentation Team \\
\textbf{Review Date}: Annual review required

\end{document}