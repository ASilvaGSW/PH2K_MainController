\documentclass[12pt,a4paper]{article}
\usepackage[utf8]{inputenc}
\usepackage{geometry}
\usepackage{longtable}
\usepackage{array}
\usepackage{graphicx}
\usepackage{hyperref}
\geometry{margin=1in}

\title{Hose Jig System\newline Technical and User Manual}
\author{Generated by Cascade AI}
\date{\today}

\begin{document}
\maketitle

\tableofcontents
\newpage

% --------------------------------------------------
\section{Introduction}
This document provides technical and user documentation for the Hose Jig system as implemented in the file \texttt{hose\_jig.ino}. It includes wiring, hardware, command protocol, expected behaviors, and operational instructions.

% --------------------------------------------------
\section{Hardware Overview}
\subsection{Microcontroller and Interfaces}
\begin{itemize}
    \item \textbf{Platform:} ESP32
    \item \textbf{CAN Buses:}
    \begin{itemize}
        \item CAN0 (TWAI, ESP32 Integrated)
        \item CAN1 (MCP2515, External)
    \end{itemize}
    \item \textbf{EEPROM:} For persistent counters and configuration
    \item \textbf{Servo Motors:} 3 (Left, Center, Right)
    \item \textbf{Linear Actuator:} X Axis
    \item \textbf{Inductive Sensors:} 3 (Left, Center, Right)
\end{itemize}

\subsection{Pin Assignments and Wiring}
\begin{longtable}{|l|l|l|}
\hline
\textbf{Function} & \textbf{Pin} & \textbf{Details} \\
\hline
CAN0 TX           & GPIO4   & ESP32 Integrated CAN TX      \\
CAN0 RX           & GPIO5   & ESP32 Integrated CAN RX      \\
CAN1 CS           & GPIO16  & MCP2515 Chip Select          \\
CAN1 INT          & GPIO17  & MCP2515 Interrupt            \\
Servo Left        & GPIO12  & PWM                          \\
Servo Center      & GPIO13  & PWM                          \\
Servo Right       & GPIO14  & PWM                          \\
Sensor Left       & GPIO26  & Inductive, INPUT	extunderscore PULLUP      \\
Sensor Center     & GPIO27  & Inductive, INPUT	extunderscore PULLUP      \\
Sensor Right      & GPIO25  & Inductive, INPUT	extunderscore PULLUP      \\
\hline
\end{longtable}

\subsection{Other Hardware Details}
\begin{itemize}
    \item Servo PWM: 50Hz, 500-2500us pulse width
    \item Linear Actuator CAN ID: 0x2CE
    \item Device CAN ID: 0x0CA
    \item Response CAN ID: 0x4CA
\end{itemize}

% --------------------------------------------------
\section{Command Cases Overview}
The system processes CAN instructions based on the first byte of incoming data. Below are all supported cases:

\begin{longtable}{|c|p{8cm}|}
\hline
\textbf{Case Code} & \textbf{Description} \\
\hline
0x01 & Reset microcontroller (restart) \\
0x02 & Ping (read X axis status) \\
0x03 & Home actuator (move to home position) \\
0x04 & Move actuator to absolute position (angle, orientation) \\
0x05 & Move all servos to open position \\
0x06 & Move all servos to close position \\
0x08 & Read servo movement counter \\
0x09 & Reset servo movement counter \\
0x0A & Move all servos to specified position \\
0x0B & Move actuator to insertion position \\
0x0C & Move actuator to insertion position (duplicate) \\
0x0D & Read actuator movement counter \\
0x0E & Reset actuator movement counter \\
0x10 & Update SERVO\_OPEN\_ANGLE \\
0x11 & Update SERVO\_CLOSE\_ANGLE \\
0x12 & Update ACTUATOR\_DELIVER\_POSITION \\
0x13 & Update ACTUATOR\_INSERTION\_POSITION \\
0x14 & Read SERVO\_OPEN\_ANGLE \\
0x15 & Read SERVO\_CLOSE\_ANGLE \\
0x16 & Read ACTUATOR\_DELIVER\_POSITION \\
0x17 & Read ACTUATOR\_INSERTION\_POSITION \\
0xFF & Power off (move all to home position) \\
Other & Unknown command (returns error) \\
\hline
\end{longtable}

% --------------------------------------------------
\section{Command Input/Output Table}
\begin{longtable}{|c|p{5cm}|p{4cm}|p{4cm}|}
\hline
\textbf{Case} & \textbf{Input Data (CAN)} & \textbf{Expected Output} & \textbf{Notes} \\
\hline
0x01 & [0x01, ...] & Device restarts, sends [0x01, 0x01, ...] & Reset MCU \\
0x02 & [0x02, ...] & [0x02, 0x01, ...] & Ping reply \\
0x03 & [0x03, ...] & [0x03, status, ...] & Home actuator, status: 0x01=OK, 0x02=Timeout, 0x04=No local network \\
0x04 & [0x04, posH, posL, orient, ...] & [0x04, status, ...] & Move actuator to position \\
0x05 & [0x05, ...] & [0x05, 0x01, ...] & Servos open \\
0x06 & [0x06, ...] & [0x06, 0x01, ...] & Servos close \\
0x08 & [0x08, ...] & [0x08, 0x01, counterH, counterL, ...] & Servo counter read \\
0x09 & [0x09, ...] & [0x09, 0x01, ...] & Servo counter reset \\
0x0A & [0x0A, pos, ...] & [0x0A, 0x01, ...] & Servos to position \\
0x0B & [0x0B, ...] & [0x0B, status, ...] & Actuator to insertion position \\
0x0C & [0x0C, ...] & [0x0C, status, ...] & Actuator to insertion position \\
0x0D & [0x0D, ...] & [0x0D, 0x01, counterH, counterL, ...] & Actuator counter read \\
0x0E & [0x0E, ...] & [0x0E, 0x01, ...] & Actuator counter reset \\
0x10 & [0x10, valH, valL, ...] & [0x10, 0x01, valH, valL, ...] & Update SERVO\_OPEN\_ANGLE \\
0x11 & [0x11, valH, valL, ...] & [0x11, 0x01, valH, valL, ...] & Update SERVO\_CLOSE\_ANGLE \\
0x12 & [0x12, valH, valL, ...] & [0x12, 0x01, valH, valL, ...] & Update ACTUATOR\_DELIVER\_POSITION \\
0x13 & [0x13, valH, valL, ...] & [0x13, 0x01, valH, valL, ...] & Update ACTUATOR\_INSERTION\_POSITION \\
0x14 & [0x14, ...] & [0x14, 0x01, high, low, ...] & Read SERVO\_OPEN\_ANGLE \\
0x15 & [0x15, ...] & [0x15, 0x01, high, low, ...] & Read SERVO\_CLOSE\_ANGLE \\
0x16 & [0x16, ...] & [0x16, 0x01, high, low, ...] & Read ACTUATOR\_DELIVER\_POSITION \\
0x17 & [0x17, ...] & [0x17, 0x01, high, low, ...] & Read ACTUATOR\_INSERTION\_POSITION \\
0xFF & [0xFF, ...] & [0xFF, status, ...], then [0xFF, 0x01, ...] & Power off, move all to home \\
Other & Any & [0xFF, ...] (error) & Error response \\
\hline
\end{longtable}

% --------------------------------------------------
\section{Technical Manual}
\subsection{System Architecture}
The Hose Jig system is built on an ESP32 microcontroller, utilizing FreeRTOS for multitasking and two CAN buses for communication: one integrated (TWAI) and one external (MCP2515). The system controls three servos and a linear actuator, with inductive sensors for feedback. EEPROM is used for storing counters and configuration.

\subsection{Communication Protocol}
\begin{itemize}
    \item CAN0 (TWAI) is used for receiving main instructions and sending responses.
    \item CAN1 (MCP2515) is used for actuator control.
    \item Each command is a CAN frame with the first data byte as the case code.
    \item Replies use a separate response CAN ID (0x4CA).
\end{itemize}

\subsection{Persistent Storage}
Counters for servo and actuator movements, as well as configuration values (angles, positions), are stored in EEPROM to retain values across power cycles.

\subsection{Error Handling}
\begin{itemize}
    \item Status codes: 0x01 (OK), 0x02 (Timeout), 0x04 (No local network)
    \item Unknown commands return an error response.
\end{itemize}

\subsection{Troubleshooting}
\begin{itemize}
    \item Ensure all wiring matches the pinout table.
    \item Check CAN bus connections and termination.
    \item If actuators/servos do not move, verify power and CAN status.
    \item Use ping (0x02) to check device responsiveness.
\end{itemize}

% --------------------------------------------------
\section{User Manual}
\subsection{Setup}
\begin{enumerate}
    \item Connect all hardware as per the wiring table.
    \item Ensure CAN bus termination resistors are present.
    \item Power the ESP32 and peripherals.
\end{enumerate}

\subsection{Operation}
\begin{enumerate}
    \item Send commands over CAN0 (TWAI) using the correct device CAN ID (0x0CA).
    \item Monitor responses on CAN0 with response CAN ID (0x4CA).
    \item Use provided cases to control servos and actuator, and to read or reset counters.
    \item For safe shutdown, use the power-off command (0xFF).
\end{enumerate}

\subsection{Maintenance}
\begin{itemize}
    \item Periodically check and reset movement counters as needed.
    \item Inspect wiring and connectors for wear or damage.
    \item Store configuration changes using the update commands (0x10--0x13).
\end{itemize}

\subsection{Safety}
\begin{itemize}
    \item Always power off before servicing hardware.
    \item Avoid operating actuators or servos with obstructions present.
    \item Ensure CAN bus and power wiring are secure.
\end{itemize}

\section{Appendix}
\subsection{Default Configuration Values}
\begin{itemize}
    \item SERVO\_OPEN\_ANGLE: 180
    \item SERVO\_CLOSE\_ANGLE: 0
    \item ACTUATOR\_DELIVER\_POSITION: 100
    \item ACTUATOR\_INSERTION\_POSITION: 100
\end{itemize}

\end{document}
