\documentclass{article}
\usepackage[utf8]{inputenc}
\usepackage{fancyhdr}
\usepackage{longtable}
\usepackage{geometry}
\usepackage{xurl}
\geometry{a4paper, margin=1in}

\pagestyle{fancy}
\fancyhf{}
\rhead{Stamper Firmware}
\lhead{PH2K New Protocol}
\cfoot{\thepage}

\title{Firmware Documentation: Stamper}
\author{Alan Silva}
\date{December 2025}

\begin{document}

\maketitle

\section{Overview}
This document provides detailed technical documentation for the \texttt{stamper.ino} firmware, a complex control system for a robotic stamping machine within the PH2K New Protocol ecosystem. The firmware runs on an ESP32 and manages a wide array of actuators, motors, and sensors.

The system's core functionality includes precise control over two linear actuators (Y and Z axes), three pump actuators, two servo motors, and two stepper motors. It also integrates two optical sensors for position feedback and uses a dual CAN bus architecture for robust communication.

\section{Hardware Architecture}
The firmware coordinates a sophisticated set of hardware components:
\begin{itemize}
    \item \textbf{Linear Actuators:} Two primary actuators for the Y and Z axes, and three additional actuators for pump control. These are managed via the local MCP2515 CAN bus.
    \item \textbf{Stepper Motors:} Two \texttt{AccelStepper} motors for the Y and Z axes, providing precise positional control.
    \item \textbf{Servo Motors:} Two \texttt{ESP32Servo} motors, identified as \texttt{servo\_cover} and \texttt{servo\_stamper}.
    \item \textbf{Optical Sensors:} Two digital sensors used for position detection and homing procedures, connected to GPIO pins.
    \item \textbf{Dual CAN Bus:}
    \begin{itemize}
        \item \textbf{TWAI (General Network):} The ESP32's integrated CAN controller (GPIO 4, 5) is used for primary command and response. The device ID is \texttt{0x005} and response ID is \texttt{0x405}.
        \item \textbf{MCP2515 (Local Network):} An external CAN controller (GPIO 26, 25) manages communication with the linear actuators.
    \end{itemize}
    \item \textbf{EEPROM:} Onboard non-volatile memory is used to store movement and activation counters for all major components.
\end{itemize}

\section{Software Architecture}
The software is built on FreeRTOS to handle the complexity of controlling multiple components concurrently.

\subsection{Core Components}
\begin{itemize}
    \item \textbf{FreeRTOS:} Manages multitasking by dedicating Core 0 to the \texttt{twai\_listener\_task} for uninterrupted CAN message reception, while the main application logic runs on Core 1.
    \item \textbf{Instruction Queue:} A FreeRTOS queue buffers incoming CAN commands, decoupling the listener task from the processing task and ensuring reliable command handling.
    \item \textbf{Libraries:} The firmware leverages several libraries, including \texttt{ESP32-TWAI-CAN}, \texttt{mcp\_can}, \texttt{AccelStepper}, \texttt{ESP32Servo}, and a custom \texttt{linear\_actuator.h} class.
\end{itemize}

\subsection{Key Functions}
\begin{description}
    \item[\texttt{setup()}] Initializes all hardware peripherals, including sensors, motors, actuators, and both CAN buses. It loads all counters from EEPROM and creates the FreeRTOS listener task.
    \item[\texttt{loop()}] The main task on Core 1. It dequeues and processes instructions via \texttt{process\_instruction()} and continuously calls the \texttt{run()} method for the stepper motors to manage their non-blocking movement.
    \item[\texttt{twai\_listener\_task()}] Runs on Core 0, listening for CAN frames on the TWAI bus. Validated frames for this device are sent to the instruction queue.
    \item[\texttt{process\_instruction(CanFrame instruction)}] The central command dispatcher. A large \texttt{switch} statement decodes the CAN command and calls the appropriate functions to control the hardware. It sends a status response after each operation.
    \item[\texttt{waitForCanReply(uint16\_t expectedId)}] A blocking function that waits for a reply from the local CAN bus (MCP2515), used to confirm actions from the linear actuators.
\end{description}

\section{CAN Command Reference}
The table below details the CAN commands handled by the \texttt{process\_instruction} function.

\begin{longtable}{|p{0.1\linewidth}|p{0.3\linewidth}|p{0.3\linewidth}|p{0.2\linewidth}|}
    \hline
    \textbf{Command} & \textbf{Description} & \textbf{Input Parameters} & \textbf{Return Value} \\
    \hline
    \endhead

    \texttt{0x01} & Reset microcontroller. & None & \url{[0x01, 0x01, ...]} \\
    \hline
    \texttt{0x02} & Heartbeat check. & None & \url{[0x02, 0x01, ...]} \\
    \hline
    \texttt{0x03} & Home Y and Z linear actuators. & None & \url{[0x03, status, ...]} \\
    \hline
    \texttt{0x04} & Move Y actuator to absolute position. & \url{data[1-2]: pos, data[3]: orientation} & \url{[0x04, status, ...]} \\
    \hline
    \texttt{0x05} & Move Z actuator to absolute position. & \url{data[1-2]: pos, data[3]: orientation} & \url{[0x05, status, ...]} \\
    \hline
    \texttt{0x06} & Read Z actuator movement counter. & None & \url{[0x06, 0x01, high, low, ...]} \\
    \hline
    \texttt{0x07} & Read Y actuator movement counter. & None & \url{[0x07, 0x01, high, low, ...]} \\
    \hline
    \texttt{0x08} & Reset Y actuator movement counter. & None & \url{[0x08, 0x01, ...]} \\
    \hline
    \texttt{0x09} & Reset Z actuator movement counter. & None & \url{[0x09, 0x01, ...]} \\
    \hline
    \texttt{0x0A} & Read Y-axis stepper movement counter. & None & \url{[0x0A, 0x01, high, low, ...]} \\
    \hline
    \texttt{0x0B} & Move Y-axis stepper to position. & \url{data[1-3]: targetPos, data[4]: dir} & \url{[0x0B, 0x01, ...]} \\
    \hline
    \texttt{0x0C} & Home Y-axis stepper. & None & \url{[0x0C, 0x01, ...]} \\
    \hline
    \texttt{0x12} & Reset Y-axis stepper movement counter. & None & \url{[0x12, 0x01, ...]} \\
    \hline
    \texttt{0x13} & Home Y axis using \texttt{go\_home}. & None & \url{[0x13, status, ...]} \\
    \hline
    \texttt{0x14} & Home Z axis using \texttt{go\_home}. & None & \url{[0x14, status, ...]} \\
    \hline
    \texttt{0x15} & Move Y actuator with speed control. & \url{data[1-2]: pos, data[3]: speed, data[4]: acc} & \url{[0x15, status, ...]} \\
    \hline
    \texttt{0x16} & Read Z-axis stepper movement counter. & None & \url{[0x16, 0x01, high, low, ...]} \\
    \hline
    \texttt{0x17} & Move Z-axis stepper to position. & \url{data[1-3]: targetPos, data[4]: dir} & \url{[0x17, 0x01, ...]} \\
    \hline
    \texttt{0x18} & Home Z-axis stepper. & None & \url{[0x18, 0x01, ...]} \\
    \hline
    \texttt{0x19} & Reset Z-axis stepper movement counter. & None & \url{[0x19, 0x01, ...]} \\
    \hline
    \texttt{0x1A} & Control pump in speed mode with timer. & \url{data[1]: pump, data[2]: dir, data[3-4]: speed, data[5]: time, data[6]: acc} & \url{[0x1A, status, ...]} \\
    \hline
    \texttt{0x1B} & Read pump movement counter. & \url{data[1]: pump} & \url{[0x1B, 0x01, high, low, ...]} \\
    \hline
    \texttt{0x1C} & Reset pump movement counter. & \url{data[1]: pump} & \url{[0x1C, 0x01, ...]} \\
    \hline
    \texttt{0x1D} & Control servo motor angle. & \url{data[1]: servo, data[2]: angle} & \url{[0x1D, 0x01, ...]} \\
    \hline
    \texttt{0x1E} & Read servo movement counter. & \url{data[1]: servo} & \url{[0x1E, 0x01, high, low, ...]} \\
    \hline
    \texttt{0x1F} & Reset servo movement counter. & \url{data[1]: servo} & \url{[0x1F, 0x01, ...]} \\
    \hline
    \texttt{0x20} & Move stepper until optical sensor is activated. & \url{data[1]: stepper, data[2]: sensor, data[3]: dir, data[4-7]: maxSteps} & \url{[0x20, status, ...]} \\
    \hline
    \texttt{0x21} & Read optical sensor activation counter. & \url{data[1]: sensor} & \url{[0x21, 0x01, high, low, ...]} \\
    \hline
    \texttt{0x22} & Reset optical sensor activation counter. & \url{data[1]: sensor} & \url{[0x22, 0x01, ...]} \\
    \hline
    \texttt{0xFF} & Power off sequence (homes all axes). & None & \url{[0xFF, status, ...]} \\
    \hline

\end{longtable}

\end{document}